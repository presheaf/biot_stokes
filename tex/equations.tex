
\documentclass{article}
\usepackage[utf8]{inputenc}
\usepackage{amsmath}
\usepackage{amssymb}
\usepackage{amsbsy}
\usepackage{graphicx}
\usepackage{xspace}
\usepackage{multicol}


\setcounter{secnumdepth}{-1}

\input{commands}


\begin{document}

\title{Equations for system}
\maketitle


We follow \cite{ambartsumyan} very very closely.

\section{Domain}
Our domain $\Omega$ is $d=2 \text{ or } 3$-dimensional, and partitioned into \stokes and \darcy, with $\interface = \stokes \cap \darcy$ being the $(d-1)$-dimensional interface. The boundary $\partial \Omega$ is bisected into $\stokesbdy = \partial \Omega \cap \partial \stokes$ and $\darcybdy = \partial \Omega \cap \partial \darcy$. We assume each region is connected, reasonably smooth and all that.

\section{Unknowns}
The unknowns of the system and the corresponding test functions are:

\begin{itemize}
\item \uf, \vf : free flow fluid velocity. Defined on \stokes.
\item \up, \vp : porous flow fluid velocity. Defined on \darcy.
\item \pf, \wf : free flow fluid pressure. Defined on \stokes.
\item \pp, \wp : porous flow fluid pressure. Defined on \darcy.
\item \disp, \disptest : displacement. Defined on \darcy.
\item \mult, \multtest : normal stress balance Lagrange multiplier. Defined on \interface. In \cite{ambartsumyan}, denoted $\lambda, \mu_h$.

\end{itemize}

\section{Parameters}

\begin{description}
\item[$\mu_f$] fluid viscosity (denoted $\mu$ in the \cite{ambartsumyan})
  
\item[$\lambda_p, \mu_p$] Lamé parameters. Denoted $\mu$ in \cite{ambartsumyan}.
\item[$\alpha$] Biot-Willis constant
\item[$K$] Permeability tensor. Symmetric, bounded, positive definite. I take it to be scalar.
\item[$\alpha_{BJS}$] Friction coefficient
\item[$s_0$] Storage coefficient
\item[$B=$] $\frac {\mu_f \alpha_{BJS}} {\sqrt{K}}$ The coefficient in the BJS condition. Could be zero.
\end{description}

\section{Notation}
\begin{itemize}
\item \nf, \np are the outward unit normal vectors to $\partial \stokes, \partial \darcy$. 
\item $\taubf_{f, j}, j = 1, \ldots, d-1$ is an orthogonal system of unit tangent vectors at \interface.
\item $\D(\mathbf{v}) \defeq \frac 12 (\grad \mathbf{v} + \grad \mathbf{v}^T)$
\item $\sigmabf_f(\uf, \pf) \defeq -\pf \mathbf{I} + 2 \mu_f \D(\uf)$
\item $\sigmabf_p(\disp, \pp) \defeq \lambda_p (\div \disp)\mathbf{I}  + 2 \mu_p \D(\disp) - \alpha \pp \mathbf{I} $
\end{itemize}

Next, here are a bunch of bilinear forms used in the problem: 
\begin{itemize}
\item $a_f(\uf, \vf) = \inner{2\mu \D(\uf)} {\D(\vf)}_{\stokes}$
\item $a_p^d(\up, \vp) = \inner{\mu K^{-1}\up} {\vp}_{\darcy}$
\item $a_p^e(\disp, \disptest) = \inner{\mu_p \D(\disp)} {\D(\disptest)}_{\darcy} + \inner{\lambda_p \div \disp} {\div \disptest}_{\darcy}$
\item $b_f(\vf, \wf) = -\inner{\div \vf}{\wf}_{\stokes}$
\item $b_p(\vp, \wp) = -\inner{\div \vp}{\wp}_{\darcy}$
  
\item $a_{BJS}(\uf, \disp; \vf, \disptest) = \frac {\mu \alpha_{BJS}} {\sqrt{K}} \sum_{j=1}^{d-1} \inner{(\uf - \disp) \cdot \taubf_j}{ (\vf - \disptest) \cdot \taubf_j}_{\interface} $
\item $b_{\interface}(\vf, \vp, \disptest; \multtest) = b_{\interface}^{\vf}(\vf, \multtest)  + b_{\interface}^{\vp}(\vp, \multtest) + b_{\interface}^{\disptest}(\disptest, \multtest)$

\item $b_{\interface}^{\vf}(\vf, \multtest) = \inner{\vf \cdot \nf}{\multtest}_{\interface}$
\item $b_{\interface}^{\vp}(\vp, \multtest) = \inner{\vp \cdot \np}{\multtest}_{\interface}$
\item $b_{\interface}^{\disptest}(\disptest, \multtest) = \inner{\disptest \cdot \np}{\multtest}_{\interface}$
  
\end{itemize}


\section{Strong formulation}

Stokes (applies in \stokes):
\begin{subequations}
  \begin{align}
    - \div \sigmabf_f (\uf, \pf) = f_f    \label{eq:stokes_stress} \\
    \div \uf = q_f    \label{eq:stokes_conservation}
  \end{align}
\end{subequations}

Darcy (applies in \darcy):
(Here, we have a source term $g_p$ not in \cite{ambartsumyan}.)
\begin{equation}
    \up = - \frac {K} {\mu} \grad \pp + g_p     \label{eq:darcy}
  \end{equation}

  
Biot (applies in \darcy):
\begin{subequations}
  \begin{align}
    - \div \sigmabf_p (\disp, \pp) = f_p     \label{eq:biot_stress} \\
    \ddt{} \left ( s_0 \pp + \alpha \div \disp \right ) + \div \up = q_p    \label{eq:biot_conservation}
  \end{align}
\end{subequations}
\section{Interface conditions}

Conservation of mass:
\begin{equation}
\uf  \cdot \nf + \left ( \ddt{\disp} + \up \right ) \cdot \np = 0  \label{eq:massconservation}
\end{equation}
Balance of stress :
\begin{equation}
  -(\sigmabf_f \nf) \cdot \nf = \pp + C_p = \lambda  \label{eq:stressbalance1_mult}
\end{equation}
\begin{equation}
  \sigmabf_f \nf + \sigmabf_p \np = C_p \np \label{eq:stressbalance2}
\end{equation}
BJS condition:
\begin{equation}
  -(\sigmabf_f \nf) \cdot \taubf_j = B \left ( \uf - \ddt{\disp} \right ) \cdot \taubf_j
  \label{eq:BJS}
\end{equation}


\cite{ambartsumyan} does not have the constant $C_p$ in (5), but their interface is 'actually flat' and not a vessel wall with muscles extering force of their own, so that's reasonable.


\section{Variational formulation}
Having used a backward Euler discretization of the time derivative, \cite{ambartsumyan} obtain the following variational formulation

\begin{subequations}
  \begin{align}
    a_f(\uf, \vf) +& b_f(\vf, \pf)  + a^e_p(\disp, \disptest) \label{eq:varform1} \\
    +&\alpha b_p(\disptest, \pp)  + a_p^d(\up, \vp) + b_p(\vp, \pp)  \nonumber \\
    + &b_{\interface}\left (\vf, \vp, \disptest; \lambda \right ) + a_{BJS}\left (\uf, \frac{\disp} {\dt}; \vf, \disptest \right)\nonumber \\
                   = a_{BJS}\left (0, \frac{\disp^{n-1}} {\dt}; \vf, \disptest \right) + & \inner{\vp}{C_p\np}_{\interface} + (\sigmabf_f\nf, \vf)_{\stokesbdy}  \nonumber \\
    + & (\sigmabf_p\np, \disptest)_{\darcybdy} + (\pp\np, \vp)_{\darcybdy} + \inner{C_p\np}{\disptest}_{\interface}\nonumber \\
    + & \inner{f_f}{\vf}_{\stokes} + \inner{f_p}{\disptest}_{\darcy}  + \inner{g_p}{\vp}_{\darcy}\nonumber \\ \nonumber \\ %
    \inner{s_0 \frac {\pp} {\dt}}{\wp}_{\darcy}  &- \alpha b_p\left ( \frac{\disp} {\dt}, \wp \right ) - b_p(\up, \wp) - b_f(\uf, \wf) \label{eq:varform2}
    \\ = \inner{s_0 \frac {\pp^{n-1}} {\dt}}{\wp}_{\darcy} &- \alpha b_p\left ( \frac {\disp^{n-1}} {\dt}, \wp \right )  + \inner{q_f}{\wf}_{\stokes} + \inner{q_p}{\wp}_{\darcy}  \nonumber \\ \nonumber \\
    b_{\interface}\left (\uf, \up, \frac {\disp} {\dt}; \multtest \right ) &= b^{\disptest}_{\interface}\left (\frac {\disp^{n-1}} {\dt}; \multtest \right ) \label{eq:varform3}
  \end{align}
\end{subequations}

Here the unknowns with no superscript mean the unknowns at time $n$ (e.g. $\uf = \uf^n$). 



\section{Derivation of weak form}
Briefly, \eqref{eq:varform1} is \eqref{eq:stokes_stress} multiplied by \vf integrated over \stokes; \eqref{eq:biot_stress} multiplied by \disptest integrated over \darcy; and \eqref{eq:darcy} multiplied \vp integrated over \darcy. The interface conditions are also all used.

To be more detailed, multiply \eqref{eq:stokes_stress} by \vf and integrate over \stokes. By standard vector calculus,
$$\intS - \vf \cdot \div \sigmabf_f = \intS \sigmabf_f \colon \grad \vf - \intSbdyI \sigmabf_f\nf  \cdot \vf.$$
Now, as the $\partial \stokes$ consists of \stokesbdy and \interface, the boundary term splits into an integral over \stokesbdy which we can move to the RHS by applying boundary conditions \footnote{Specifically, Dirichlet conditions on \uf or Neumann conditions on $\sigmabf_f$.}, and an interface term $-I_{\vf} = -\intI \sigmabf_f \nf \cdot \vf$.

Before proceeding, we expand\footnote{The equality $\D(\uf) \colon \grad \vf = \D(\uf) \colon \D(\vf)$ looks false because $\grad \vf \neq \D(\vf)$, but $\D(\uf)$ is symmetric, so $\D(\uf) \colon \grad \vf = \D(\uf) \colon \grad \vf^T = \D(\uf) \colon \D(\vf)$.}
\begin{align*}
  \intS \sigmabf_f \colon \grad \vf &= \intS (-\pf \mathbf{I} + 2 \mu_f \D(\uf)) \colon \grad \vf \\
                                    &= \intS -\pf \div \vf + \intS 2 \mu_f \D(\uf) \colon \grad \vf \\
                                      &= b_f(\vf, \pf) + a_f(\uf, \vf)
\end{align*}
So the contribution here is $$a_f(\uf, \vf) + b_f(\vf, \pf) -I_{\vf} - (\sigmabf_f\nf, \vf)_{\stokesbdy}.$$


% Next, multiply \eqref{eq:stokes_conservation} by \wf and integrate over \stokes. This becomes
% $$\intS \wf \cdot \div \uf = $$

Next, multiply \eqref{eq:biot_stress} by \disptest and integrate over \darcy. By exactly the same argument,
$$\intD - \disptest \cdot \div \sigmabf_p = \intD \sigmabf_p \colon \grad \disptest - \intDbdyI \sigmabf_p\np  \cdot \disptest.$$

Again, the boundary term splits into an integral over \darcybdy where we need boundary conditions \footnote{Specifically, Dirichlet conditions on \disp or Neumann conditions on $\sigmabf_p$.} and an interface term $-I_{\disptest} = -\intI \sigmabf_p \np \cdot \disptest$. Expanding,

\begin{align*}
  \intD \sigmabf_p \colon \grad \disptest &=  \intD  \left ( \lambda_p (\div \disp)(\div \disptest)  + 2 \mu_p \D(\disp) \colon \grad \disptest \right)
                                            - \alpha \pp \div \disptest \\
                                          &= a^e_p(\disp, \disptest) + \alpha b_p(\disptest, \pp)
\end{align*}
So the contribution is $$a^e_p(\disp, \disptest) + \alpha b_p(\disptest, \pp) - I_{\disptest} - (\sigmabf_p\np, \disptest)_{\darcybdy}.$$

Next, multiply \eqref{eq:darcy} by $\frac {\mu} {K} \vp$ and integrate over \darcybdy. Integration by parts yields
$$ \intD \frac {\mu} {K} \vp \cdot \uf -  \intD \pp \div \vp = \intDbdyI \pp \vp \cdot \np$$ 
The boundary term splits into an integral over \darcybdy where we need boundary conditions \footnote{Specifically, Dirichlet conditions on \up or Neumann conditions on \pp.} and an interface term $I_{\vp} = \intI \pp \vp \cdot \np$, so the contribution is
$$a_p^d(\up, \vp) + b_p(\vp, \pp) + I_{\vp} - (\pp \np, \vp)_{\darcybdy}.$$

Next, once we add all these equations together, we will have to handle the sum of the interface terms $$-I_{\vf} - I_{\disptest} + I_{\vp} = -\intI \sigmabf_f \nf \cdot \vf -\intI \sigmabf_p \np \cdot \disptest + \intI \pp \vp \cdot \np$$

We start with $\intI \pp \vp \cdot \np$. By \eqref{eq:stressbalance1_mult}, $\pp = \mult - C_p$ on \interface, so
$$I_{\vp} = \intI \pp \vp \cdot \np = \intI \mult \vp \cdot \np - \intI C_p \vp \cdot \np$$

The second term can then go live on the right hand side. Next, we treat the other two interface terms. By \eqref{eq:stressbalance2}, $\sigmabf_p \np = - \sigmabf_f \nf + C_p \nf$, so we have that $$I_{\vf} + I_{\disptest} = \intI (\vf - \disptest) \cdot \sigmabf_f \nf + \intI C_p \disptest \cdot  \nf$$
Again, the second term can go on the right hand side. Next, note that the BJS condition \eqref{eq:BJS} gives us information on the tangential component of $\sigmabf_f \nf$, while \eqref{eq:stressbalance1_mult} gives us information on the normal component. As $\nf, \taubf_1, \taubf_2$ form an orthonormal system, we have that\footnote{This is just use of the fact when a vector $\mathbf{v}$ is written in a basis $\mathbf{e}_i$, the coefficient of $\mathbf{e}_i$    }
\begin{align*}
  \vf - \disptest = \: &\nf ((\vf - \disptest) \cdot \nf) + \sum_j \taubf_j ((\vf - \disptest) \cdot \taubf_j) \\
  \Longrightarrow \hspace{0.02\textwidth}  (\sigmabf_f \nf) \cdot (\vf - \disptest) = & \: ((\sigmabf_f \nf) \cdot \nf) \: ((\vf - \disptest) \cdot \nf) \\
  & + \sum_j ((\sigmabf_f \nf) \cdot \taubf_j) ((\vf - \disptest) \cdot \taubf_j)
\end{align*}
By \eqref{eq:stressbalance1_mult}, $(\sigmabf_f \nf) \cdot \nf = -\lambda$, and by \eqref{eq:BJS}, $(\sigmabf_f \nf) \cdot \taubf_j = -B \left ( \uf - \ddt{\disp} \right ) \cdot \taubf_j$, so

\begin{align*}
(\sigmabf_f \nf) \cdot (\vf - \disptest) =  \: &-\lambda\:  ((\vf - \disptest) \cdot \nf) \\
  &- \sum_j \left (B \left ( \uf - \ddt{\disp} \right ) \cdot \taubf_j \right ) ((\vf - \disptest) \cdot \taubf_j)
\end{align*}

We can then use that $\nf = - \np$ to write $\lambda \: (\vf - \disptest) \cdot \nf = \lambda \: (\vf  \cdot \nf + \disptest \cdot \np)$.
Putting it all together,

\begin{align*}
  -(I_{\vf} + I_{\disptest}) + I_{\vp}  =& \intI \lambda\: (\vf  \cdot \nf + (\disptest + \vp) \cdot \np) \\
  + &\intI \left (B \left ( \uf - \ddt{\disp} \right ) \cdot \taubf_j \right ) ((\vf - \disptest) \cdot \taubf_j) \\
  - &\intI C_p \vp \cdot \np \\
  = b_{\interface}\left (\vf, \vp, \disptest; \lambda \right ) & + a_{BJS}\left (\uf, \ddt{\disp}; \vf, \disptest \right) - \inner{\vp}{C_p\np}_{\interface} \\
\end{align*}

We have now derived all the necessary identities, and summing them all yields the following:
\begin{align*}
  a_f(\uf, \vf) +& b_f(\vf, \pf)  + a^e_p(\disp, \disptest)  \\ +&\alpha b_p(\disptest, \pp)  + a_p^d(\up, \vp) + b_p(\vp, \pp)  \\
  + &b_{\interface}\left (\vf, \vp, \disptest; \lambda \right ) + a_{BJS}\left (\uf, \ddt{\disp}; \vf, \disptest \right)\\
  = \inner{\vp}{C_p\np}_{\interface} +& (\sigmabf_f\nf, \vf)_{\stokesbdy} + (\sigmabf_p\np, \disptest)_{\darcybdy} \\
  + &(\pp\np, \vp)_{\darcybdy} + \inner{C_p\np}{\disptest}_{\interface}\\
\end{align*}

If \ddt{\disp} is now discretized by a backward Euler difference, this is exactly \eqref{eq:varform1}, if we remember that we ignored the source term and thus need to add it.

The next two are not as bad. \eqref{eq:varform2} is just \eqref{eq:stokes_conservation} multiplied by \wf integrated over \stokes plus \eqref{eq:biot_conservation} multiplied by \wp integrated over Darcy.

Doing the above (no integration of parts needed) yields

$$s_0\inner{\ddt{\pp}}{\wp}_{\darcy} - \alpha b_p(\ddt{\disp}, \wp)
- b_p(\up, \wp) - b_f(\uf, \wf)$$ 

Once the \ddt{}'s are discretized by a backward Euler difference, this is exactly \eqref{eq:varform2}.

Finally, \eqref{eq:varform3} is obtained by taking \eqref{eq:massconservation}, multiplying by \multtest and integrating over \interface. This yields $$b_{\interface} (\uf, \up, \ddt{\disp} ;\multtest) = 0$$
which, when \ddt{\disp} is discretized using a Backward Euler difference, yields \eqref{eq:varform3}.



\section{Matrix form}
Denote by $\matrixform{c(\mathbf{u}, \mathbf{v})}$ the matrix/vector of the bilinear/linear form $c$, meaning the matrix with entries $c(\mathbf{e}^U_i, \mathbf{e}^V_j)$, where the test function varies as you move down the rows. Then swapping each argument between test and trial function transposes the matrix, so for example $$\matrixform{b_f(\vf, \pf)} = \matrixform{b_f(\uf, \wf)}^T. $$
 
We write our problem in a matrix form, but before doing so, we take a look at what's going on at the interface, as two of our bilinear forms are mixed. First, note that
\begin{align*}
a_{BJS}(\uf, \disp; \vf, \disptest) = B \sum_{j=1}^{d-1} & \inner{\uf \cdot \taubf_j}{\vf \cdot \taubf_j}_{\interface} - \inner{\disp \cdot \taubf_j}{\vf \cdot \taubf_j}_{\interface} \\
                                                         & - \inner{\uf \cdot \taubf_j}{\disptest \cdot \taubf_j}_{\interface}+ \inner{\disp \cdot \taubf_j}{\disptest \cdot \taubf_j}_{\interface}
\end{align*}
splits into $S_{\vf, \uf} - S_{\vf, \disp} - S_{\disptest, \uf} + S_{\disptest, \disp}$
where $S_{\mathbf{a}, \mathbf{b}} = B \sum_{j=1}^{d-1} \inner{\mathbf{b} \cdot \taubf_j}{\mathbf{a} \cdot \taubf_j}_{\interface}$ (see the Thoughts-section for how this might be conveniently implemented)

Next, looking at the other interface term, $b_{\interface}(\vf, \vp, \disptest; \mult)$, note that it splits into sums of mass matrices of normal components. More precisely,

$$b_{\interface}(\vf, \vp, \disptest; \mult) = N^f_{\vf, \mult} + N^p_{\vp, \mult} + N^p_{\disptest, \mult} $$

where $N^*_{\mathbf{a}, b} = \inner{\mathbf{a} \cdot \nporf}{b}_{\interface}$. With the exception of the RHS term $\inner{\vp}{C_p\np}_{\interface}$, which is easier because it only lives in one function space, these are the only interface terms we need to assemble.

So on the implementation side, the only matrices we need to assemble on the interface are $\inner{\mathbf{a} \cdot \nporf}{b}_{\interface}$ and  $B \sum_{j=1}^{d-1} \inner{\mathbf{b} \cdot \taubf_j}{\mathbf{a} \cdot \taubf_j}_{\interface}$

So on the implementation side, the only matrices we need to assemble on the interface are $\inner{\mathbf{a} \cdot \nporf}{b}_{\interface}$ and  $B \sum_{j=1}^{d-1} \inner{\mathbf{b} \cdot \taubf_j}{\mathbf{a} \cdot \taubf_j}_{\interface}$

Now, abuse notation to let $N_*, S_*$ instead mean the matrices of the bilinear forms defined above, and define a couple more:
\begin{center}
  \begin{multicols}{2}
    Matrices
\begin{description}
\item[$A_f$] \defeq \matrixform{a_f(\uf, \vf)}
\item[$A^d_p$]\defeq  \matrixform{a_p^d(\up, \vp)}
\item[$A^e_p$]\defeq  \matrixform{a_p^e(\disp, \disptest)}
\item[$B_f$]\defeq  \matrixform{b_f(\vf, \pf)}
\item[$B^{\vp}_p$]\defeq  \matrixform{b_p(\vp, \pp)}
\item[$B^{\disptest}_p$]\defeq  \matrixform{b_p(\disptest, \pp)}
\item[$M_*$] \defeq mass matrix of unknown $* $
\end{description}
\columnbreak
Vectors
\begin{description}
\item[$L^{\sigmabf_f}$] \defeq \matrixform{\inner{\sigmabf_f \nf}{\vf}_{\stokesbdy}}
\item[$L^{\sigmabf_p}$] \defeq \matrixform{\inner{\sigmabf_p \np}{\disptest}_{\darcybdy}}
\item[$L^{\pp}$] \defeq \matrixform{\inner{\pp \np}{\vp}_{\darcybdy}}
\item[$L_{\vp}^{\Delta p}$] \defeq \matrixform{\inner{C_p\np}{\vp}_{\interface}}
\item[$L_{\disptest}^{\Delta p}$] \defeq \matrixform{\inner{C_p\nf}{\disptest}_{\interface}}
\item[$F_f, F_p, Q_f, Q_p, G_p$] \defeq \matrixform{\text{source terms}}
\item[$\uf, \pf, \ldots$] \defeq vector of dofs
  
\end{description}

\end{multicols}
\end{center}
Note that although some of these vectors involve the unknowns, but appear on the RHS, they will be known because of BCs.


Then the matrix of our system looks like
\[ \renewcommand{\arraystretch}{1.8}
  \left[\begin{array}{@{}ccc|cc|c@{}}
          % \up & \pp & \disp & \uf & \pf & \mult \\
          
          A^d_p & B^{\vp}_p & \matzero & \matzero & \matzero & N^p_{\vp, \mult} \\ % \vp -row
          - (B^{\vp}_p)^T & \frac {s_0}{\dt} M_{\pp} & - \frac{\alpha}{\dt} (B^{\disptest}_p)^T & \matzero & \matzero & \matzero\\ % \wp -row 
          \matzero & \alpha B^{\disptest}_p & A^e_p + S_{\disptest, \disp} & - S_{\disptest, \uf} & \matzero & N^p_{\disptest, \mult}\\ % \disptest -row
          \hline
          \matzero & \matzero & - \frac 1 {\dt} S_{\vf, \disp} & A_f + S_{\vf, \uf} & B_f & N^f_{\vf, \mult}\\ % \vf -row
          \matzero & \matzero & \matzero & -B_f^T & \matzero & \matzero \\ % \wf -row
          \hline
          (N^p_{\vp, \mult})^T & \matzero & \frac 1 {\dt} (N^p_{\disptest, \mult})^T & (N^f_{\vf, \mult})^T & \matzero & \matzero \\ % \multtest -row

        \end{array}\right]
      \left[\begin{array}{@{}c@{}}
                \up \\
                \pp \\
                \disp \\
                \uf \\
                \pf \\
                \mult
            \end{array}\right]
          = \newline
                      \left[\begin{array}{@{}c@{}}
                L^{\pp} + L_{\vp}^{\Delta p} + G_p\\
                              b^{\pp}  + Q_p \\
                              L^{\sigma_p} + L_{\disptest}^{\Delta p} + F_p\\
                L^{\sigma_f} + F_f\\
                Q_f \\
                \frac 1 {\dt} (N^p_{\disptest, \mult})^T \disp^{n-1} \\
        \end{array}\right]
    \]


where
$$b^{\pp} = \frac {s_0}{\dt} M_{\pp} \pp^{n-1} - \frac {\alpha} {\dt} (B_p^{\disptest})^T \disp^{n-1}$$

By scaling, this seems like it could be made symmetric. Specifically, dividing the equation for \disptest by \dt, flipping the signs of the equations for \wf and \wp gets you there:
\[ \renewcommand{\arraystretch}{1.8}
  \left[\begin{array}{@{}ccc|cc|c@{}}
          % \up & \pp & \disp & \uf & \pf & \mult \\
          
          A^d_p & B^{\vp}_p & \matzero & \matzero & \matzero & N^p_{\vp, \mult} \\ % \vp -row
          (B^{\vp}_p)^T & -\frac {s_0}{\dt} M_{\pp} & \frac{\alpha}{\dt} (B^{\disptest}_p)^T & \matzero & \matzero & \matzero\\ % \wp -row 
          \matzero & \frac {\alpha} {\dt} B^{\disptest}_p & \frac 1 {\dt}A^e_p + \frac 1 {\dt} S_{\disptest, \disp} & -\frac 1 {\dt} S_{\disptest, \uf} & \matzero &\frac 1 {\dt} N^p_{\disptest, \mult}\\ % \disptest -row
          \hline
          \matzero & \matzero & - \frac 1 {\dt} S_{\vf, \disp} & A_f + S_{\vf, \uf} & B_f & N^f_{\vf, \mult}\\ % \vf -row
          \matzero & \matzero & \matzero & B_f^T & \matzero & \matzero \\ % \wf -row
          \hline
          (N^p_{\vp, \mult})^T & \matzero & \frac 1 {\dt} (N^p_{\disptest, \mult})^T & (N^f_{\vf, \mult})^T & \matzero & \matzero \\ % \multtest -row

        \end{array}\right]
      \left[\begin{array}{@{}c@{}}
                \up \\
                \pp \\
                \disp \\
                \uf \\
                \pf \\
                \mult
            \end{array}\right]
          = \newline
                      \left[\begin{array}{@{}c@{}}
                L^{\pp} + L^{\Delta p}  + G_p\\
                              -b^{\pp} - Q_p\\
                              \frac 1 {\dt} \left ( L^{\sigma_p} + L_{\disptest}^{\Delta p} + F_p \right )\\
                L^{\sigma_f} + F_f\\
                - Q_f\\
                \frac 1 {\dt} (N^p_{\disptest, \mult})^T \disp^{n-1} \\
        \end{array}\right]
\]
This makes the $(\wp, \pp)$ block negative, though.


\section{Boundary conditions}
As the derivation of the weak form shows, I need :
\begin{itemize}
\item Dirichlet conditions on \uf or Neumann conditions on
  $\sigmabf_f$ on \stokesbdy
\item Dirichlet conditions on \disp or Neumann %
  conditions on $\sigmabf_p$ on \darcybdy
\item Dirichlet conditions on
  \up or Neumann conditions on \pp on \darcybdy.
\end{itemize}


In their numerical experiment (section 7.2), \cite{ambartsumyan} use the domain shown in figure \ref{fig:ambartsumyandomain}
\begin{figure}[h]
  \centering
  \includegraphics[width=0.15\textwidth]{img/ambartsumyandomain.png}
  \label{fig:ambartsumyandomain}
  \caption{Darcy domain from \cite{ambartsumyan}. Stokes domain is the removed 'finger'.}
\end{figure}
The Darcy boundary \darcybdy is partitioned into the left part $\darcybdy^{\text{L}}$ and the remainder $\darcybdy^{\neg \text{L}}$ in the obvious way. Physically, I think $\darcybdy^{\text{L}}$ is above ground and the other part is below ground or something.

As boundary conditions, they use:
\begin{itemize}
\item $\uf = 10 \nf$ on \stokesbdy
\item $\up \cdot \np = 0$ on $\darcybdy^{\text{left}}$
\item $\pp = 1000$ on $\darcybdy^{\neg \text{left}}$ (maybe this should be on $\darcybdy^{\text{left}}$ instead?)
\item $\disp \cdot \np = 0$ on $\darcybdy^{\neg \text{left}}$
\item $(\sigmabf_p \np) \cdot \taubf_p = 0$ on $\darcybdy^{\neg \text{left}}$


 
\end{itemize}



\section{Thoughts}
\begin{itemize}
\item Kent offered very gentle scepticism about using a 3-field formulation, and suggested not having \up as an unknown, using $\up = \grad \pp$ to remove it. I thought \cite{ambartsumyan} had some opinion on this, but on closer reading I can't find it, so maybe that's from another article. I should read up on this.

\item Why is there a 2 in front of $\mu$ in equation \eqref{eq:stokes_stress}? If $\mu$ is just divided by 2 that's fine, but then I need to divide my choice by $\mu$ accordingly.
\item As $\nf, \taubf_1, \taubf_2$ form an orthonormal system, $\sum \inner{\mathbf{b} \cdot \taubf_j}{\mathbf{a} \cdot \taubf_j}$ is the normal inner product $\inner{\mathbf{b}}{\mathbf{a}}$ minus the term $(\mathbf{a} \cdot \np) \cdot (\mathbf{b} \cdot \np)$ So it can be thought of as the inner product $\inner{\Pi{\mathbf{a}}}{\Pi{\mathbf{b}}}$ where $\Pi\mathbf{a}$ is the projection of the vector $\mathbf{a}$ onto the interface, meaning that the BJS matrix is essentially a ``mass matrix with a component missing''.

  So if you don't want to mess with tangents, the relation $$\inner{\Pi\mathbf{a}}{\Pi\mathbf{b}} = \inner{\mathbf{a}}{\mathbf{b}} - \inner{(\mathbf{a} \cdot \nf)}{(\mathbf{b} \cdot \nf)}$$ means you're set as long as you can keep track of the interface normal. (In principle it should not even need to have a consistent orientation, but in practice I imagine it might have to.) 
 
\end{itemize}


\bibliographystyle{amsalpha}
\begin{thebibliography}{99}
\bibitem{ambartsumyan}
{\sc Ambartsumyan, Ilona, et al. }, {\em "A Lagrange multiplier method for a Stokes-Biot fluid-poroelastic structure interaction model." }, arXiv preprint arXiv:1710.06750 (2017).
  
\end{thebibliography}

\end{document}
